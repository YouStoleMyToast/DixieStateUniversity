%
% Assignment 1b for CS3530 Computational Theory:
% Finite Automata
% Fall 2014
%
% Problems taken from Sipser
%

\documentclass{article}

\usepackage[margin=1in]{geometry}
\usepackage{amsfonts}
\usepackage{amsmath}
\usepackage[english]{babel}
\usepackage[utf8]{inputenc}
\usepackage{ae,aecompl}
\usepackage{emp,ifpdf}
\usepackage{graphicx}

\ifpdf\DeclareGraphicsRule{*}{mps}{*}{}\fi

\empprelude{input boxes; input theory}

% skip for paragraphs, don't indent
\addtolength{\parskip}{0.5\baselineskip}
\parindent=0pt

\begin{document}
\begin{empfile}

\begin{center}
\textbf{\Large CS 3530: Assignment 1b} \\[2mm]
Fall 2014
\end{center}

Name: Eric Beilmann

\raggedright

\section*{Exercises}

Note: In each of the following, you should show and describe the
simpler DFAs as well as the final NFA that you construct. You must
follow the steps of each construction precisely. Do not take
shortcuts or simplify the results. You do not need to show
intermediate steps.

If a DFA is called for, an NFA is not acceptable. Be sure to include
\textit{all} states of a DFA.

\subsection*{Exercise 1.4c (7 points)}

\subsubsection*{Problem}

Each of the following languages is the intersection of two simpler
languages. In each part, construct DFAs for the simpler languages,
then combine them using the construction discussed in footnote~3
(page~46) to give the state diagram of a DFA for the language given.
In all parts $\Sigma=\{a,b\}$.

\begin{itemize}
\item[c.] Language: $\{w:w$ has an even number of $a$'s and one or
two $b$'s$\}$
\end{itemize}

\subsubsection*{Solution DFAs}

\begin{center}
\begin{emp}(0,0)
	a := 2cm;
	b := a*2;
	c := a*3;
	d := a*4;
	node.q0(); q0.c = origin;
	node.q1(); q1.c = q0.c + (a,0);
	node.q2(); q2.c = q0.c + (b,0);
	node.q3(); q3.c = q0.c + (c,0);
	
	makestart(q0);
	makefinal(q1); makefinal(q2);

	drawboxed(q0,q1,q2,q3);

	edge(q0,q1,right,"b");
	edge(q1,q2,right,"b");
	edge(q2,q3,right,"b");
	
	loop(q3,right,"b");

\end{emp}
\end{center}

\begin{center}
\begin{emp}(0,0)
	a := 2cm;
	b := a*2;
	c := a*3;
	d := a*4;
	node.q0(); q0.c = origin;
	node.q1(); q1.c = q0.c + (a,0);


	
	makestart(q0);
	makefinal(q0); 

	drawboxed(q0,q1);

	edge(q0,q1,right,"a");
	edge(q1,q0,right,"");



\end{emp}
\end{center}

\subsubsection*{Solution Intersection}

\begin{center}
\begin{emp}(0,0)
	a := 2cm;
	b := a*2;
	c := a*3;
	d := a*4;
	node.q0(); q0.c = origin;
	node.q1(); q1.c = q0.c + (a,0);
	node.q2(); q2.c = q0.c + (b,0);
	node.q3(); q3.c = q0.c + (c,0);
	node.q4(); q4.c = q0.c + (0,-a);
	node.q5(); q5.c = q0.c + (a,-a);
	node.q6(); q6.c = q0.c + (b,-a);
	node.q7(); q7.c = q0.c + (c,-a);
	
	makestart(q0);
	makefinal(q1); makefinal(q2);

	drawboxed(q0,q1,q2,q3,q4,q5,q6,q7);

	edge(q0,q1,right,"b");
	edge(q1,q2,right,"b");
	edge(q2,q3,right,"b");
	
	edge(q0,q4,right,"a");
	edge(q1,q5,right,"a");
	edge(q2,q6,right,"a");
	edge(q3,q7,right,"a");
	edge(q4,q0,right,"");
	edge(q5,q1,right,"");
	edge(q6,q2,right,"");	
	edge(q7,q3,right,"");
	
	edge(q4,q5,right,"b");
	edge(q5,q6,right,"b");
	edge(q6,q7,right,"b");

	loop(q3,right,"b");
	loop(q7,right,"b");
\end{emp}
\end{center}

\subsection*{Exercise 1.5c (7 points)}

\subsubsection*{Problem}

Each of the following languages is the complement of a simpler
language. In each part, construct a DFA for the simpler language,
then use it to give the state diagram of a DFA for the language
given. In all parts $\Sigma=\{a,b\}$.

\begin{itemize}
\item[c.] Language: $\{w:w$ contains neither the substrings $ab$ nor $ba\}$
\end{itemize}

\subsubsection*{Solution}

\begin{center}
\begin{emp}(0,0)
	a := 2cm;
	b := a/2;
	c := a*2;
	node.q0(); q0.c = origin;
	node.q1(); q1.c = q0.c + (a,b);
	node.q2(); q2.c = q0.c + (c,b);
	node.q3(); q3.c = q0.c + (a,-b);
	node.q4(); q4.c = q0.c + (c,-b);
	
	
	makestart(q0);
	makefinal(q0); makefinal(q1); makefinal(q3);
	drawboxed(q0,q1,q2,q3,q4);

	edge(q0,q1,right,"a");
	edge(q1,q2,right,"b");
	edge(q0,q3,right,"b");
	edge(q3,q4,right,"a");
	
	loop(q1,up,"a");
	loop(q3,down,"b");
	loop(q2,right,"a,b");
	loop(q4,right,"a,b");
	
\end{emp}
\end{center}

\subsection*{Exercise 1.7c (6 points)}

\subsubsection*{Problem}

Give state diagrams of NFAs with the specified number of states
recognizing each of the following languages. In all parts the
alphabet is $\{0,1\}$.

\begin{itemize}
\item[c.] Language: $\{w:w$ contains an even number of $0$'s, or
contains exactly two $1$'s$\}$ with six states
\end{itemize}

\subsubsection*{Solution}

\begin{center}
\begin{emp}(0,0)
	a := 2cm;
	b := a*2;
	c := a*3;

	node.q0(); q0.c = origin;
	node.q1(); q1.c = q0.c + (a,0);
	node.q2(); q2.c = q0.c + (b,0);
	node.q3(); q3.c = q0.c + (0,-a);
	node.q4(); q4.c = q0.c + (a,-a);
	node.q5(); q5.c = q0.c + (b,-a);
	
	makestart(q0);
	makefinal(q0); makefinal(q1); makefinal(q2); makefinal(q5);
	drawboxed(q0,q1,q2,q3,q4,q5);

	edge(q0,q1,right,"1");
	edge(q1,q2,right,"1");
	
	edge(q0,q3,right,"0");
	edge(q3,q0,left,"");
	edge(q1,q4,right,"0");
	edge(q4,q1,left,"");
	edge(q2,q5,right,"0");
	edge(q5,q2,left,"");
	
	edge(q3,q4,right,"1");
	edge(q4,q5,right,"1");
	
\end{emp}
\end{center}

\end{empfile}
\immediate\write18{mpost -tex=latex \jobname}
\end{document}



