%
% Assignment 6a for CS3530 Computational Theory:
% Turing Machines
% Fall 2014
%
% Problems taken from Sipser
%

\documentclass{article}

\usepackage[margin=1in]{geometry}
\usepackage{amsfonts}
\usepackage{amsmath}
\usepackage[english]{babel}
\usepackage[utf8]{inputenc}
\usepackage{ae,aecompl}
\usepackage{enumerate}

% skip for paragraphs, don't indent
\parskip 6pt plus 1pt
\parindent=0pt
\raggedbottom

% a list environment with no bullets or numbers
\newenvironment{indentlist}{\begin{list}{}{\addtolength{\itemsep}{0.5\baselineskip}}}{\end{list}}

\begin{document}
\begin{center}
\textbf{\Large CS 3530: Assignment 6a} \\[2mm]
Fall 2014
\end{center}

\raggedright

\section*{Exercises}

\subsection*{Problem 3.8c (20 points)}

\subsubsection*{Problem}

Give implementation-level descriptions of Turing machines that
decide the following languages over the alphabet $\{0,1\}$. Note:
Sipser describes the differences between \textit{formal
descriptions}, \textit{implementation-level descriptions}, and
\textit{high-level descriptions} on pages 156--157.

\begin{enumerate}
\item[\bfseries c.] $\{w:w$ does not contain exactly twice as many
$0$s as $1$s$\}$ 
\end{enumerate}

\subsubsection*{Solution}
M = on string w: \\
1. Scan the tape and mark the first unmarked 1. \\ \parindent=12pt
   if no unmarked 1s are found $accept$. \\
   else go to step 2. \\ \ \\ \parindent=0pt

2. Scan the tape and mark the first unmarked 0. \\ \parindent=12pt
   if no unmarked 0s are found $accept$. \\
   else go to step 3. \\ \ \\ \parindent=0pt

3. Scan the tape and mark the first unmarked 0. \\ \parindent=12pt
   if no unmarked 0s are found $accept$. \\
   else go to step 4. \\ \ \\ \parindent=0pt

4. Scan the tape and mark the first unmarked 1. \\ \parindent=12pt
   if no unmarked 1s are found go to step 5. \\ 
   else go to step 2. \\ \ \\ \parindent=0pt

5. Scan the tape and mark the first unmarked 0. \\ \parindent=12pt
   if no unmarked 0s are found $reject$. \\
   else $accept$. \\ \ \\ \parindent=0pt

\end{document}









