%
% Assignment 4c for CS3530 Computational Theory:
% Context-free Grammars and Pushdown Automata
% Fall 2014
%
% Problems taken from Sipser
%

\documentclass{article}

\usepackage[margin=1in]{geometry}
\usepackage{amsfonts}
\usepackage{amsmath}
\usepackage[english]{babel}
\usepackage[utf8]{inputenc}
\usepackage{ae,aecompl}
\usepackage{emp,ifpdf}
\usepackage{graphicx}
\usepackage{enumerate}

\ifpdf\DeclareGraphicsRule{*}{mps}{*}{}\fi

\empaddtoTeX{\usepackage{amsmath}}
\empprelude{input boxes; input theory}

% skip for paragraphs, don't indent
\parskip 6pt plus 1pt
\parindent=0pt
\raggedbottom

% a list environment with no bullets or numbers
\newenvironment{indentlist}{\begin{list}{}{\addtolength{\itemsep}{0.5\baselineskip}}}{\end{list}}

\begin{document}
\begin{empfile}

\begin{center}
\textbf{\Large CS 3530: Assignment 4c} \\[2mm]
Fall 2014
\end{center}

\raggedright

\section*{Problems}

\subsection*{Problem 2.24 (10 points)}

\subsubsection*{Problem}

Let $E=\{a^i b^j: i\neq j$ and $2i\neq j\}$. Show that $E$ is a
context-free language.

If you give a CFG, describe the role that each rule performs as well
as giving the actual rule. If you give a PDA, describe how it works
as well as giving the state diagram.

\subsubsection*{Solution}
The top path any number of a's that is greater than b
\begin{center}
\begin{emp}(0,0)
	a := 3cm;
	b := a*2;
	c := a*3;
	d := a*4;
	e := a*5;
	f := a*6;
	x := a/2;
	y := 2cm;
	k := 3.5cm;
	node.q0("q0"); q0.c = origin;
	node.q1("q1"); q1.c = q0.c + (a,0);
	node.q2("q2"); q2.c = q0.c + (b,0);
	
	node.q3("q3"); q3.c = q0.c + (c,x);
	node.q4("q4"); q4.c = q0.c + (d,x);

	node.q6("q6"); q6.c = q0.c + (c,-x);
	node.q7("q7"); q7.c = q0.c + (d,-x);

	node.q10("q10"); q10.c = q0.c + (e,0);
	
	smallnodes; node.q100(); q100.c = q0.c + (b,y); bignodes;
	smallnodes; node.q300(); q300.c = q0.c + (d,a); bignodes;
	smallnodes; node.q600(); q600.c = q0.c + (c,-a); bignodes;
	
	makestart(q1);
	makefinal(q10);

	drawboxed(q1,q2,q3,q4,q6,q7,q10,q100,q300,q600);


	edge(q1,q2,right,btex $(\varepsilon,\varepsilon)\rightarrow \$ $ etex);
	edge(q2,q100,40,btex $(a,\varepsilon)\rightarrow a $ etex);
	edge(q100,q2,40,btex $(a,\varepsilon)\rightarrow \varepsilon $ etex);
	
	edge(q2,q3,right,btex $(\varepsilon,\varepsilon)\rightarrow \varepsilon $ etex);
	edge(q3,q4,right,btex $(a,\varepsilon)\rightarrow \varepsilon $ etex);
	loop(q2,down,btex $(a,\varepsilon)\rightarrow a $ etex);

	loop(q3,up,btex $(a,\varepsilon)\rightarrow \varepsilon $ etex);
	loop(q4,down,btex $(b,a) \rightarrow \varepsilon $ etex);
	edge(q4,q300,40,btex $(b,a)\rightarrow \varepsilon $ etex);
	edge(q300,q4,40,btex $(b,\varepsilon)\rightarrow \varepsilon $ etex);
	
	edge(q2,q6,left,btex $(\varepsilon,\varepsilon)\rightarrow \varepsilon $ etex);
	edge(q6,q7,right,btex $(b,\varepsilon)\rightarrow \varepsilon $ etex);

	loop(q6,up,btex $\hspace*{1cm}(b,a) \rightarrow \varepsilon $ etex);
	loop(q7,down,btex $(b,\varepsilon) \rightarrow \varepsilon $ etex);

	edge(q6,q600,-40,btex $(b,a)\rightarrow \varepsilon $ etex);
	edge(q600,q6,-40,btex $(b,\varepsilon)\rightarrow \varepsilon $ etex);

	edge(q4,q10,left,btex $(\varepsilon,\$)\rightarrow \$ $ etex);
	edge(q7,q10,right,btex $(\varepsilon,\$)\rightarrow \$ $ etex);
	
\end{emp}
\end{center}


\newpage


\subsection*{Problem 2.44 (10 points)}

\subsubsection*{Problem}

If $A$ and $B$ are languages, define $A\diamond B=\{xy:x\in A$ and
$y\in B$ and $|x|=|y|\}$. Show that if $A$ and $B$ are regular
languages, then $A\diamond B$ is a CFL.

Note: a formal proof is not necessary; a detailed description of a
suitable construction will suffice.

\subsubsection*{Solution}

RE for A: $(aaa)^m$ (number of $a$'s is divisable by 3) $\newline$
RE for B: $b^n$ (any number of $b$'s)$\newline$

The PDA produces strings of the regular languages that fit the requirment of
having equal lengths.

\begin{center}
\begin{emp}(0,0)
	a := 3cm;
	b := a*2;
	x := 1.5cm;
	node.q0("q0"); q0.c = origin;
	node.q1("q1"); q1.c = q0.c + (a,0);
	node.q2("q2"); q2.c = q0.c + (b,x);
	node.q3("q3"); q3.c = q0.c + (b,-x);
	node.q4("q4"); q4.c = q0.c + (a,-a-x);
	node.q5("q5"); q5.c = q0.c + (0,-a-x);

	
	makestart(q0);
	makefinal(q5);

	drawboxed(q0,q1,q2,q3,q4,q5);

	edge(q0,q1,right,btex $(\varepsilon,\varepsilon)\rightarrow \$ $ etex);
	edge(q1,q2,left,btex $(a,\varepsilon)\rightarrow a $ etex);
	edge(q2,q3,left,btex $(a,\varepsilon)\rightarrow a $ etex);
	edge(q3,q1,left,btex $(a,\varepsilon)\rightarrow a $ etex);
	edge(q1,q4,right,btex $(\varepsilon,\varepsilon)\rightarrow \varepsilon $ etex);
	edge(q4,q5,right,btex $(\varepsilon,\$)\rightarrow \varepsilon $ etex);
	
	loop(q4,right,btex $(b,a)\rightarrow \varepsilon $ etex);

\end{emp}
\end{center}


\end{empfile}
\immediate\write18{mpost -tex=latex \jobname}
\end{document}



