%
% Assignment 7e for CS3530 Computational Theory:
% Computability
% Fall 2014
%
% Problems taken from Sipser
%

\documentclass{article}

\usepackage[margin=1in]{geometry}
\usepackage{amsfonts}
\usepackage{amsmath}
\usepackage[english]{babel}
\usepackage[utf8]{inputenc}
\usepackage{ae,aecompl}
\usepackage{enumerate}

% skip for paragraphs, don't indent
\parskip 6pt plus 1pt
\parindent=0pt
\raggedbottom

% a list environment with no bullets or numbers
\newenvironment{indentlist}{\begin{list}{}{\addtolength{\itemsep}{0.5\baselineskip}}}{\end{list}}

\begin{document}
\begin{center}
\textbf{\Large CS 3530: Assignment 7e} \\[2mm]
Fall 2014
\end{center}

\raggedright

\subsection*{Problem 5.14 (20 points)}

\subsubsection*{Problem}

Consider the problem of determining whether a Turing machine $M$ on
an input $w$ ever attempts to move its head left when its head is on
the left-most tape cell. Formulate this problem as a language and
show that it is undecidable. You may not use Rice's Theorem.
(note: do by reduction(halting problem))
\subsubsection*{Solution}

%We will describe the above TM as: $L_{TM}$ = On input $w$ if head is at $q_0$ and attempts to move head left. 
$L_{TM}$ = On input $w$ if head is at $q_0$ and attempts to move head left then it accepts. \\
$HALT_{TM}$ = \{$\langle M, w\rangle$| $M$ is a $TM$ and $M$ halts on input $w$ \} \\
Since we $HALT_{TM}$ is undecidable we will show that $L_{TM}$ is undecidable by doing a 
reduction where $HALT_{TM}$ $\leq$ $L_{TM}$. \\ \ \\

We will assume construct $S_{TM}$ to decide $HALT_{TM}$ \\
$S$="On input $\langle M, w\rangle$: \\
\hspace{5 mm}	1. Construct $T_{TM}$ from $M$ where $T$ marks the left-most cell and shifts the head to the right \\
\hspace{5 mm}	2. $T_{TM}$ runs $M$ on $w$: \\
\hspace{10 mm}		1. If $T$ is on the marked cell run $M$ on $w$, \\
\hspace{15 mm}			Then $T$ moves to the right simulating $M$ reaching the left-most cell \\
\hspace{10 mm}		2. If $M$ halts and accepts, \\
\hspace{15 mm}			Then $T$ moves its head all the way to the left \\
\hspace{5 mm}	3. If $T$ accepts then $S$ accpets \\
\hspace{5 mm}	4. Else $T$ rejects then $S$ rejects \\ \ \\

Since $T$ moves its head off the left-most cell when $M$ halts. Then $S$ decides $HALT$.
Since $HALT$ is undecidable and $T$ decides $HALT$ then $L$ is undecidable.

\end{document}





