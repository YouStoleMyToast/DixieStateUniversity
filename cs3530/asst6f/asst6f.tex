%
% Assignment 6f for CS3530 Computational Theory:
% Turing Machines
% Fall 2014
%
% Problems taken from Sipser
%

\documentclass{article}

\usepackage[margin=1in]{geometry}
\usepackage{amsfonts}
\usepackage{amsmath}
\usepackage[english]{babel}
\usepackage[utf8]{inputenc}
\usepackage{ae,aecompl}
\usepackage{enumerate}

% skip for paragraphs, don't indent
\parskip 6pt plus 1pt
\parindent=0pt
\raggedbottom

% a list environment with no bullets or numbers
\newenvironment{indentlist}{\begin{list}{}{\addtolength{\itemsep}{0.5\baselineskip}}}{\end{list}}

\begin{document}
\begin{center}
\textbf{\Large CS 3530: Assignment 6f} \\[2mm]
Fall 2014
\end{center}

\raggedright

\section*{Exercises}

\subsection*{Exercise 5.4 (20 points)}

\subsubsection*{Problem}

If $A\leq_m B$ and $B$ is a regular language, does that imply that
$A$ is a regular language? Why or why not?

\subsubsection*{Solution}

No, $A$ is not implicitly a regular language.
The relationship between $A$ and $B$ indicates they are both
decidable or undecidable. Since $B$ is regular then they are both decidable
but that doesn't tell you if $A$ is regualar. In fact it is possible to do a 
reduction on a non-regular language and get a regular language. \\ \ \\ 

Example: \\
$A$ = \{a$^n$b$^n$: $n$ $\ge$ 0\} and $B$ = \{b*\} \\ 
The reduction function that reduces $A$ to $B$ will reject $w$ 
if it starts with the letter b ($w \notin A$), \\
otherwise it will accept it ($w \in A$). \\



\null
\vfill

Definitions \\
computable function:\\
A function $f$: $\Sigma^* \rightarrow \Sigma^*$ is a computable function
if some turing machine $M$, on every input $w$, halts with just $f(w)$ on its 
tape.\\ \ \\

mapping reducable: \\
if: there is a function $f$: $\Sigma^* \rightarrow \Sigma^*$
where for every $w \in A \leftrightarrow f(w) \in B$. \\
then: $f$ is the reduction of $A$ to $B$.

\end{document}














