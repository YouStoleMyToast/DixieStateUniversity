\documentclass[letterpaper,11pt]{article}

\usepackage{amsfonts}
\usepackage{amsmath}
\usepackage[english]{babel}
\usepackage[utf8]{inputenc}
\usepackage{ae,aecompl}

% allow metapost figures to be included inline
% note: you must invoke latex using:
%
%     pdflatex -shell-escape <inputfile>
%
% to allow it to invoke external commands. See the very end of the
% document as well.
\usepackage{emp,ifpdf}
\usepackage{graphicx}

% convert metapost figures to .eps or .pdf automatically when
% including them
\ifpdf\DeclareGraphicsRule{*}{mps}{*}{}\fi

% include the metapost macros
\empprelude{input boxes; input theory}

% skip for paragraphs, don't indent
\addtolength{\parskip}{0.5\baselineskip}
\parindent=0pt

\begin{document}

% create a metapost file for figures to be dumped to.  It's easiest
% to use a single file for all the figures in the document
\begin{empfile}

\begin{center}
\textbf{\LARGE Finite Automata using Metapost Macros} \\
Russ Ross
\end{center}

\section{Simple Example}

Here is a DFA that accepts strings over $\Sigma=\{a,b\}$ with an
even number of $a$'s and an even number of $b$'s. Using ``curve''
curves the edge slightly to the left:

% center the figure
\begin{center}

% Start a new figure. The (0,0) indicates the size of the figure.
% Those dimensions are available as the variables w and h in
% metapost so you can make scalable figures. I usually just ignore
% them, hence (0,0).
\begin{emp}(0,0)
  % it's good practice to pick a basic size unit and make all your
  % distances relative to that unit
  u := 2cm;

  % create a node, and position its center at the origin
  node.q0(); q0.c = origin;

  % create 3 more nodes, and position their respective centers
  % relative to the center of the first node.
  node.q1(); q1.c = q0.c + (u,0);
  node.q2(); q2.c = q0.c + (0,-u);
  node.q3(); q3.c = q0.c + (u,-u);

  % mark q0 as a start node
  makestart(q0);

  % mark q3 as an accept node
  makefinal(q3);

  % draw the nodes
  drawboxed(q0,q1,q2,q3);

  % an edge from q0 to q1, curved 25 degrees, with label $a$
  edge(q0,q1,25,A);

  % additional edges with default curve and $a$ label
  edge(q1,q0,curve,A);
  edge(q2,q3,curve,A);
  edge(q3,q2,curve,A);

  % additional edges with default curve and $b$ label
  edge(q0,q2,curve,B);
  edge(q2,q0,curve,B);
  edge(q1,q3,curve,B);
  edge(q3,q1,curve,B);
\end{emp}
\end{center}

\section{Curved Edges}

This DFA was constructed from an NFA using the subset construction.
It has straight edges, edges with custom curves, and loops. For
curved edges, the number indicates the angle of the curve in
degrees. More specifically, it gives the number of degrees of the
point on the circle where the edge should emerge from the node. An
edge with angle zero would emerge directly toward the target
node\footnote{This doesn't actually work, because it can't figure
out where to put the label. For a straight edge, use ``left'' and
``right'' to tell it which side to put the label on.}; anything else
is added or subtracted from that angle.

\begin{center}
\begin{emp}(0,0)
  % Pick a size of nodes other than the default (bignodes). This
  % also resets the edge color, solid/dashed, etc.
  mediumnodes;
  u := 1cm;

  % create a node with a zero inside it, centered at the origin
  node.a("0"); a.c = (0,0);

  % position additional nodes using absolute coordinates
  node.b("1"); b.c = (u,0);
  node.c("2"); c.c = (2u,0);
  node.d("03"); d.c = (3u,0);
  node.e("01"); e.c = (4u,0);
  node.f("12"); f.c = (5u,0);
  node.g("02"); g.c = (6u,0);

  % labels can use LaTeX format with btex ... etex
  node.h(btex $\emptyset$ etex); h.c = (0,-u);
  makestart(a); makefinal(a,d,e,g);

  edge(a,b,right,A);
  edge(b,c,right,A);
  edge(c,d,right,B);
  edge(d,e,right,A);
  edge(e,f,right,A);
  edge(f,g,right,A);
  edge(a,h,right,B);

  % negative angles emerge curved to the left instead of right
  edge(c,a,-45,A);
  edge(g,e,-45,A);
  edge(g,d,-60,B);
  edge(f,d,60,B);

  loop(h,down,A);
  loop(h,left,B);

  drawboxed(a,b,c,d,e,f,g,h);
\end{emp}
\end{center}

\section{Colored Edges}

This NFA has red edges when there are multiple transitions for a
single input letter. When using straight edges, put ``left'' or
``right'' to indicate where the label should go:

\begin{center}
\begin{emp}(0,0)
  smallnodes;
  node.a(); a.c = (0,0);
  makestart(a); makefinal(a);

  % this time we didn't define a scaling unit, so we use standard
  % units instead
  node.b(); b.c = (1.5cm,1cm);
  node.c(); c.c = (2cm,0);
  node.d(); d.c = (1.5cm,-1cm);

  % switch to drawing red edges
  rededges;
  edge(a,b,left,A);

  % back to black edges
  blackedges;
  edge(b,c,left,A);
  edge(c,d,left,B);
  edge(d,a,left,A);

  node.e(); e.c = (-1cm,1cm);

  % metapost is capable of quite a bit of math, so you can compute
  % positions instead of specifying literal coordinates
  node.f(); f.c = (0cm, sqrt(2)*1cm);
  rededges;
  edge(a,e,left,A);
  blackedges;
  edge(e,f,left,A);
  edge(f,a,left,B);

  node.g(); g.c = (-1cm,-1cm);
  node.h(); h.c = (0cm, -sqrt(2)*1cm);
  rededges;
  edge(a,h,left,A);
  blackedges;
  edge(h,g,left,A);
  edge(g,a,left,A);

  drawboxed(a,b,c,d,e,f,g,h);
\end{emp}
\end{center}

\section{Undirected Graphs}

You can do undirected edges and dashed edges as well. Note that
using the label ``BLANK'' skips the label on an edge. The same works
for a node, but you can also just leave the label argument blank.
For straight, unlabeled edges, You can also use ``sedge'', which
draws a simple, straight edge with no label.

\begin{center}
\begin{emp}(0,0)
  mediumnodes;
  picture x;
  u := 1cm;

  % turn off arrow heads on the edges
  undirectededges;
  node.x(X); x.c=(0,0);

  % up is a unit vector, scaling it gives it the desired length
  node.a(A); a.c=up scaled u;

  % here we scale and rotate a vector to specify the position
  node.b(B); b.c=up rotated 120 scaled u;
  node.c(C); c.c=up rotated -120 scaled u;
  drawboxed(x,a,b,c);

  % straight edges can use this shortcut
  sedge(x,a);
  sedge(x,b);
  sedge(x,c);
  dashededges;
  sedge(a,b);
  sedge(b,c);
  sedge(a,c);
  solidedges; directededges;
\end{emp}
\end{center}

\end{empfile}

% this invokes metapost on the figures. You must run latex a second
% time for the figures to be included.
\immediate\write18{mpost -tex=latex \jobname}

\end{document}
