%
% Assignment 7b for CS3530 Computational Theory:
% Computability
% Fall 2014
%
% Problems taken from Sipser
%

\documentclass{article}

\usepackage[margin=1in]{geometry}
\usepackage{amsfonts}
\usepackage{amsmath}
\usepackage[english]{babel}
\usepackage[utf8]{inputenc}
\usepackage{ae,aecompl}
\usepackage{enumerate}

% skip for paragraphs, don't indent
\parskip 6pt plus 1pt
\parindent=0pt
\raggedbottom

% a list environment with no bullets or numbers
\newenvironment{indentlist}{\begin{list}{}{\addtolength{\itemsep}{0.5\baselineskip}}}{\end{list}}

\begin{document}
\begin{center}
\textbf{\Large CS 3530: Assignment 7b} \\[2mm]
Fall 2014
\end{center}

\raggedright

\section*{Problems}

\subsection*{Problem 4.19 (20 points)}

\subsubsection*{Problem}

Let $S=\{\langle M\rangle: M$ is a DFA that accepts $w^\mathcal{R}$
whenever it accepts $w\}$. Show that $S$ is decidable.

\subsubsection*{Solution}

First construct a DFA N that accepts $w$. 
Then construct a DFA P (which is M$^\mathcal{R}$)
that accepts the reverse of the DFA M by flipping the transitions and 
the start and accept states. Then we need to determine if the DFAs N and P 
are in $\langle M\rangle$. We can determine if N and P are in 
$\langle M\rangle$ by determining the symmetric difference between the DFAs 
in M and N or P to determine they are equivalent.



\end{document}




