%
% Assignment 1c for CS3530 Computational Theory:
% Finite Automata
% Fall 2014
%
% Problems taken from Sipser
%

\documentclass{article}

\usepackage[margin=1in]{geometry}
\usepackage{amsfonts}
\usepackage{amsmath}
\usepackage[english]{babel}
\usepackage[utf8]{inputenc}
\usepackage{ae,aecompl}
\usepackage{emp,ifpdf}
\usepackage{graphicx}

\ifpdf\DeclareGraphicsRule{*}{mps}{*}{}\fi

\empprelude{input boxes; input theory}

% skip for paragraphs, don't indent
\addtolength{\parskip}{0.5\baselineskip}
\parindent=0pt

\begin{document}
\begin{empfile}

\begin{center}
\textbf{\Large CS 3530: Assignment 1c} \\[2mm]
Fall 2014
\end{center}

Name: Eric Beilmann

\raggedright

\section*{Exercises}

\subsection*{Exercise 1.8b (3 points)}

\subsubsection*{Problem}

Use the construction given in the proof of Theorem~1.45 to give the
state diagrams of NFAs recognizing the union of the languages given.

\begin{itemize}
\item[b.] Language: $L_1\cup L_2$ where $L_1$ is the language from
1.6c and $L_2$ is the language from 1.6f \\ (note: both language are
from assignment 1a)

Language from 1.6c: $\{w:w$ contains the substring $0101$, i.e.,
$w=x0101y$ for some $x$ and $y\}$

Language from 1.6f: $\{w:w$ doesn't contain the substring $110\}$
\end{itemize}

\subsubsection*{Solution}

\begin{center}
\begin{emp}(0,0)
	a := 2cm;
	b := a*2;
	c := a*3;
	d := a*4;
	e := a/2;
	node.q0(); q0.c = origin;
	node.q1(); q1.c = q0.c + (a,0);
	node.q2(); q2.c = q0.c + (b,0);
	node.q3(); q3.c = q0.c + (c,0);
	node.q4(); q4.c = q0.c + (d,0);
	node.q5(); q5.c = q0.c + (0,-a);
	node.q6(); q6.c = q0.c + (a,-a);
	node.q7(); q7.c = q0.c + (b,-a);
	node.q8(); q8.c = q0.c + (c,-a);
	
	node.q9(); q9.c = q0.c + (-a,-e);
	
	makestart(q9);
	makefinal(q4);makefinal(q5);makefinal(q6);makefinal(q7);

	drawboxed(q0,q1,q2,q3,q4,q5,q6,q7,q8,q9);
	edge(q9,q0,right,EMPTYSTRING);
	edge(q9,q5,right,EMPTYSTRING);

	edge(q0,q1,right,"0");
	edge(q1,q2,right,"1");
	edge(q2,q3,right,"0");
	edge(q3,q4,right,"1");
	edge(q2,q0,-curve,"1");
	edge(q3,q1,-curve,"0");
	loop(q0,down,"1");
	loop(q1,down,"0");
	loop(q4,down,"0,1");

	edge(q5,q6,curve,"1");
	edge(q6,q5,curve,"0");
	edge(q6,q7,right,"1");
	edge(q7,q8,right,"0");
	loop(q5,down,"0");
	loop(q6,down,"1");
	loop(q7,down,"0,1");

\end{emp}
\end{center}

\newpage

\subsection*{Exercise 1.9b (3 points)}

\subsubsection*{Problem}

Use the construction given in the proof of Theorem~1.47 to give the
state diagrams of NFAs recognizing the concatenation of the
languages given.

\begin{itemize}
\item[b.] Language: $L_1\circ L_2$ where $L_1$ is the language from
1.6b and $L_2$ is the language from 1.6m \\ (note: both language are
from assignment 1a)

Language from 1.6b: $\{w:w$ contains at least three $1$s$\}$

Language from 1.6m: The empty set
\end{itemize}

\subsubsection*{Solution}


\begin{center}
\begin{emp}(0,0)
	a := 2cm;
	b := a*2;
	c := a*3;
	d := a*4;
	node.q0(); q0.c = origin;
	node.q1(); q1.c = q0.c + (a,0);
	node.q2(); q2.c = q0.c + (b,0);
	node.q3(); q3.c = q0.c + (c,0);
	node.q4(); q4.c = q0.c + (d,0);

	makestart(q0);
	makefinal(q3);

	drawboxed(q0,q1,q2,q3,q4);

	edge(q0,q1,right,"1");
	edge(q1,q2,right,"1");
	edge(q2,q3,right,"1");
	edge(q3,q4,right,EMPTYSTRING);
	
	loop(q0,down,"0");
	loop(q1,down,"0");
	loop(q2,down,"0");
	loop(q3,down,"0,1");
	loop(q4,down,"0,1");




\end{emp}
\end{center}







\newpage

\subsection*{Exercise 1.15 (7 points)}

\subsubsection*{Problem}

Give a counterexample to show that the following construction fails
to prove Theorem~1.49 
\footnote{Theorem~1.49: The class of regular languages is closed under the star operation.}
, the closure of the class of regular languages under the star operation.%
\footnote{In other words, you must present a finite automaton,
$N_1$, for which the constructed automaton $N$ does not recognize
the star of $N_1$'s language.}
Let $N_1=(Q_1,\Sigma,\delta_1,q_1,F_1)$ recognize $A_1$. Construct
$N=(Q_1,\Sigma,\delta,q_1,F)$ as follows. $N$ is supposed to
recognize $A^*_1$.



\begin{enumerate}
\renewcommand{\labelenumi}{\alph{enumi}}
\item The states of $N$ are the states of $N_1$.
\item The start state of $N$ is the same as the start state of $N_1$.
\item $F=\{q_1\}\cup F_1$.

The accept states $F$ are the old accept states plus its start state.
\item Define $\delta$ so that for any $q\in Q$ and any
$a\in\Sigma_\varepsilon$,
$$
\delta(q,a)=
\begin{cases}
\delta_1(q,a) & q\notin F_1\text{ or }a\neq\varepsilon \\
\delta_1(q,a)\cup\{q_1\} & q\in F_1\text{ and }a=\varepsilon.
\end{cases}
$$
\end{enumerate}
(Suggestion: Show this construction graphically, as in Figure~1.50.)

\subsubsection*{Solution Original}
Language: $\{w:w$ contains at least three $1$s$\}$

\begin{center}
\begin{emp}(0,0)
	a := 2cm;
	b := a*2;
	c := a*3;
	d := a*4;
	node.q0(); q0.c = origin;
	node.q1(); q1.c = q0.c + (a,0);
	node.q2(); q2.c = q0.c + (b,0);
	node.q3(); q3.c = q0.c + (c,0);

	makestart(q0);
	makefinal(q3);

	drawboxed(q0,q1,q2,q3);

	edge(q0,q1,right,"1");
	edge(q1,q2,right,"1");
	edge(q2,q3,right,"1");
	
	
	loop(q0,down,"0");
	loop(q1,down,"0");
	loop(q2,down,"0");
	loop(q3,down,"0,1");

\end{emp}
\end{center}


\subsubsection*{Solution Modified}
$\Sigma$  = $\{0\}$ could be a solution.

\begin{center}
\begin{emp}(0,0)
	a := 2cm;
	b := a*2;
	c := a*3;
	d := a*4;
	node.q0(); q0.c = origin;
	node.q1(); q1.c = q0.c + (a,0);
	node.q2(); q2.c = q0.c + (b,0);
	node.q3(); q3.c = q0.c + (c,0);

	makestart(q0);
	makefinal(q0);makefinal(q3);

	drawboxed(q0,q1,q2,q3);

	edge(q0,q1,right,"1");
	edge(q1,q2,right,"1");
	edge(q2,q3,right,"1");
	
	
	loop(q0,down,"0");
	loop(q1,down,"0");
	loop(q2,down,"0");
	loop(q3,down,"0,1");

\end{emp}
\end{center}




\newpage

\subsection*{Exercise 1.16 (7 points)}

\subsubsection*{Problem}

Use the construction given in Theorem~1.39 to convert the following
two nondeterministic finite automata to equivalent deterministic
finite automata.

\begin{center}
\begin{tabular}{cc}
\begin{emp}(0,0)
  bignodes;
  u := 1.5cm;
  node.a1("1"); a1.c=(0,0);
  node.a2("2"); a2.c=(0,-u);
  makestart(a1); makefinal(a1);
  drawboxed(a1,a2);
  edge(a1,a2,curve,AB);
  edge(a2,a1,curve,B);
  loop(a1,right,A);
\end{emp}
&
\qquad\begin{emp}(0,0)
  bignodes;
  u := 1.5cm;
  s := u*sqrt(2);
  node.b1("1"); b1.c=(0,0);
  node.b2("2"); b2.c=(s,0);
  node.b3("3"); b3.c=(.5s,-u);
  makestart(b1); makefinal(b2);
  drawboxed(b1,b2,b3);
  edge(b1,b2,curve,E);
  edge(b2,b1,curve,A);
  edge(b1,b3,right,A);
  edge(b3,b2,right,AB);
  loop(b3,right,B);
\end{emp}
\\
\textbf{(a)} & \qquad\textbf{(b)}
\end{tabular}
\end{center}

\subsubsection*{Solution}

\begin{center}
\begin{tabular}{cc}
\begin{emp}(0,0)

  a := 2cm;
  node.a1("1"); a1.c=(0,0);
  node.a2("2"); a2.c=(0,-a);
  
  makestart(a1); makefinal(a1);
  drawboxed(a1,a2);
  
  edge(a1,a2,curve,B);
  edge(a2,a1,curve,B);
  loop(a1,right,A);
  loop(a2,right,A);
  
\end{emp}
&
\qquad\begin{emp}(0,0)

  a := 2cm;
  b := 2*a;
  c := 3*a;
  d := 4*a;

  node.b0("0");       b0.c=(a,b);
  %node.b1("1");       b1.c=(c,b);
  node.b2("2");       b2.c=(a,a);
  node.b3("3");       b3.c=(c,a);
  node.b12("1,2");    b12.c=(a,0);
  node.b13("1,3");    b13.c=(c,-a);
  node.b23("2,3");    b23.c=(c,0);
  node.b123("1,2,3"); b123.c=(a,-a);

  
  makestart(b12); 
  makefinal(b2);makefinal(b12);makefinal(b23);makefinal(b123);
  drawboxed(b0,b2,b3,b12,b13,b23,b123);
  
  loop(b0,up,"a,b");        
  %edge(b1,b3,right,A);    edge(b1,b0,right,B);
  edge(b2,b12,right,A);   edge(b2,b0,right,B);
  edge(b3,b2,right,A);    edge(b3,b23,right,B);
  edge(b12,b23,right,A);  edge(b12,b3,right,B);
  edge(b13,b23,curve,A);  edge(b13,b23,-curve,B);
  edge(b23,b123,right,A); edge(b23,b2,right,B);
  edge(b123,b12,right,A); loop(b123,right,B);
  
  
\end{emp}
\\
\textbf{(a)} & \qquad\textbf{(b)}
\end{tabular}
\end{center}

\end{empfile}
\immediate\write18{mpost -tex=latex \jobname}
\end{document}
