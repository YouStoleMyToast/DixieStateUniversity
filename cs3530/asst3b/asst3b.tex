%
% Assignment 3b for CS3530 Computational Theory:
% Regular Languages
% Fall 2014
%
% Problems taken from Sipser
%

\documentclass{article}

\usepackage[margin=1in]{geometry}
\usepackage{amsfonts}
\usepackage{amsmath}
\usepackage[english]{babel}
\usepackage[utf8]{inputenc}
\usepackage{ae,aecompl}
\usepackage{emp,ifpdf}
\usepackage{graphicx}
\usepackage{enumerate}
\usepackage{parskip}

\ifpdf\DeclareGraphicsRule{*}{mps}{*}{}\fi

\empaddtoTeX{\usepackage{amsmath}}
\empprelude{input boxes; input theory}

% skip for paragraphs, don't indent
\parskip 6pt plus 1pt
\parindent=0pt
\raggedbottom
\newenvironment{indentlist}{\begin{list}{}{\addtolength{\itemsep}{0.5\baselineskip}}}{\end{list}}

\begin{document}
\begin{empfile}

\begin{center}
\textbf{\Large CS 3530: Assignment 3b} \\[2mm]
Fall 2014
\end{center}

\raggedright

\section*{Exercises}

\subsection*{Exercise 2.9 (10 points)}

\subsubsection*{Problem}

Give a context-free grammar that generates the language
$$
A=\{a^ib^jc^k:i=j\text{ or }j=k\text{ where }i,j,k\geq 0\}.
$$

For all CFGs, describe the role that each rule performs as well as
giving the actual rule.

\subsubsection*{Solution}

\setlength{\parindent}{89pt}

{\noindent S -> $\varepsilon$ | T | U}  {\indent We will need to first 
choose if i=j or j=k} 

\setlength{\parindent}{11pt} 

{\noindent T -> $\varepsilon$ | TaTbT | TbTaT | TcT} {\indent This will allow any 
combination of i=j with any number of k}

\setlength{\parindent}{7pt} 

{\noindent U -> $\varepsilon$ | UbUcU | UcUbU | UaU} {\indent This will allow any 
combination of j=k with any number of i}

\setlength{\parindent}{0pt}

\section*{Problems}

\subsection*{Problem 1.47 (10 points)}

\subsubsection*{Problem}

Let $\Sigma=\{1,\#\}$ and let

$$ Y=\{w:w=x_1\#x_2\#\cdots\#x_k\text{ for }k\geq0\text{, each }x_i\in1^*\text{, and }x_i\neq x_j\text{ for }i\neq j\} $$

Prove that $Y$ is not regular.

\subsubsection*{Solution}

Assume Y is regular \newline
string s=1$^p$\#1$^p$ \newline
split s into s=xyz where: \newline
1. xy$^i$z $\in$ Y(for each $i$>=0) \newline
2. |y| > 0 \newline
3. |xy| <= p \newline

Any value of p would result in the same number of ones on both sides of the \# 
any other format of string s could result in two different amounts of 1's but that 
would result in future $x \in Y$ to have the same number of 1's

%Y = (Q, $\Sigma$, $\delta$, $q_0$, F) \newline 
%Q = \{0, 1, ... , n\} \newline
%$\Sigma$ = \{1,\#\} \newline
%$q_0 = \{0\}$ \newline 
%F = \{n\} \newline 
%$\delta$ = \newline

%$\delta (q_n,0) = q_n$ for all $n$ \newline
%$\delta (q_n,1) = q_n$ for $n_{max}$ \newline
%$\delta (q_n,1) = q_{n+1}$ for all $n < n_{max}$\newline
%$\delta (q_n,\varepsilon) = q_{n+1}$ for all $n < n_{max}$

\end{empfile}
\immediate\write18{mpost -tex=latex \jobname}
\end{document}





