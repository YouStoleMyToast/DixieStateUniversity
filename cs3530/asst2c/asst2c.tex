%
% Assignment 2c for CS3530 Computational Theory:
% Regular Expressions
% Fall 2014
%
% Problems taken from Sipser
%

\documentclass{article}

\usepackage[margin=1in]{geometry}
\usepackage{amsfonts}
\usepackage{amsmath}
\usepackage[english]{babel}
\usepackage[utf8]{inputenc}
\usepackage{ae,aecompl}
\usepackage{emp,ifpdf}
\usepackage{graphicx}
\usepackage{enumerate}

\ifpdf\DeclareGraphicsRule{*}{mps}{*}{}\fi

\empprelude{input boxes; input theory}

% skip for paragraphs, don't indent
\parskip 6pt plus 1pt
\parindent=0pt
\raggedbottom
\newenvironment{indentlist}{\begin{list}{}{\addtolength{\itemsep}{0.5\baselineskip}}}{\end{list}}

\begin{document}
\begin{empfile}

\begin{center}
\textbf{\Large CS 3530: Assignment 2c} \\[2mm]
Fall 2014
\end{center}

\raggedright

\section*{Problems}

\subsection*{Problem 1.31 (10 points)}

\subsubsection*{Problem}

For any string $w=w_1w_2\cdots w_n$, the \textbf{\textit{reverse}}
of $w$, written $w^\mathcal{R}$, is the string $w$ in reverse order,
$w_n\cdots w_2w_1$. For any language $A$, let
$A^\mathcal{R}=\{w^\mathcal{R}: w\in A\}$. Show that if $A$ is
regular, so is $A^\mathcal{R}$.

\subsubsection*{Solution}
\subsubsection*{non-reversed DFA}

\begin{center}
\begin{emp}(0,0)
	a := 2cm;
	b := a*2;

	node.q0(); q0.c = origin;
	node.q1(); q1.c = q0.c + (a,0);
	node.q2(); q2.c = q0.c + (b,0);


	makestart(q0);
	makefinal(q2);

	drawboxed(q0,q1,q2);

	edge(q0,q1,right,"1");
	edge(q1,q2,right,"1");
	
	
	loop(q0,down,"0");
	loop(q1,down,"0");
	loop(q2,down,"0,1");

\end{emp}
\end{center}

\subsubsection*{DFA reversed into NFA}

\begin{center}
\begin{emp}(0,0)
	a := 2cm;
	b := a*2;

	node.q0(); q0.c = origin;
	node.q1(); q1.c = q0.c + (a,0);
	node.q2(); q2.c = q0.c + (b,0);


	makestart(q0);
	makefinal(q2);

	drawboxed(q0,q1,q2);

	edge(q0,q1,right,"1");
	edge(q1,q2,right,"1");
	
	
	loop(q2,down,"0");
	loop(q1,down,"0");
	loop(q0,down,"0,1");

\end{emp}
\end{center}

\newpage

\subsection*{Problem 1.43 (10 points)}

\subsubsection*{Problem}

Let $A$ be any language. Define \textit{DROP-OUT}$(A)$ to be the
language containing all strings that can be obtained by removing one
symbol from a string in $A$. Thus, \textit{DROP-OUT}$(A)=\{xz:
xyz\in A$ where $x,z\in\Sigma^*,y\in\Sigma\}$. Show that the class
of regular languages is closed under the \textit{DROP-OUT}
operation. Give both a proof by picture and a more formal proof by
construction as in Theorem~1.47.

\subsubsection*{Solution Description}


N = (Q, $\Sigma$, $\delta$, $q_0$, F) \newline 
Q = \{0, 1, ... , n\} \newline
$\Sigma$ = \{0,1\} \newline
$q_0 = \{0\}$ \newline 
F = \{n\} \newline 
$\delta$ = \newline

$\delta (q_n,0) = q_n$ for all $n$ \newline
$\delta (q_n,1) = q_n$ for $n_{max}$ \newline
$\delta (q_n,1) = q_{n+1}$ for all $n < n_{max}$\newline
$\delta (q_n,\varepsilon) = q_{n+1}$ for all $n < n_{max}$

\subsubsection*{Solution Diagram}

\begin{center}
\begin{emp}(0,0)
	a := 2cm;
	b := a*2;

	node.q0("0"); q0.c = origin;
	node.q1("1"); q1.c = q0.c + (a,0);
	node.q2("n"); q2.c = q0.c + (b,0);


	makestart(q0);
	makefinal(q2);

	drawboxed(q0,q1,q2);

	edge(q0,q1,right,"1");
	edge(q1,q2,right,"1");
	
	edge(q0,q1,45, E );
	edge(q1,q2,45, E );
	
	loop(q0,down,"0");
	loop(q1,down,"0");
	loop(q2,down,"0,1");

\end{emp}
\end{center}


\end{empfile}
\immediate\write18{mpost -tex=latex \jobname}
\end{document}
