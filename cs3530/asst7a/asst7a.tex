%
% Assignment 7a for CS3530 Computational Theory:
% Computability
% Fall 2014
%
% Problems taken from Sipser
%

\documentclass{article}

\usepackage[margin=1in]{geometry}
\usepackage{amsfonts}
\usepackage{amsmath}
\usepackage[english]{babel}
\usepackage[utf8]{inputenc}
\usepackage{ae,aecompl}
\usepackage{enumerate}

% skip for paragraphs, don't indent
\parskip 6pt plus 1pt
\parindent=0pt
\raggedbottom

% a list environment with no bullets or numbers
\newenvironment{indentlist}{\begin{list}{}{\addtolength{\itemsep}{0.5\baselineskip}}}{\end{list}}

\begin{document}
\begin{center}
\textbf{\Large CS 3530: Assignment 7a} \\[2mm]
Fall 2014
\end{center}

\raggedright

\section*{Problems}

\subsection*{Problem 3.19 (20 points)}

\subsubsection*{Problem}
Show that every infinite Turing-recognizable language has an infinite decidable subset. \\
hint: think about enumerators
\subsubsection*{Solution}

If language $L$ is recognizable it must have an enumerator $E$. 
We will then set up the enumerator $E'$ as having a counter $c$ that 
keeps track of the size of strings. \\ \ \\

So as $E'$ simulates $E$ it processes individual strings $w$. \\
If the length of $w$ is greater than $c$, then print $w$ and set $c$ 
equal to the length of $w$. \\
Else do nothing with $w$. \\ \ \\

So $E'$ will print strings if they are longer than the preceding ones(in lexicographic order)
so $E'$ will only print strings in $L$. 
Since $L$ is infinite $E'$ will print an infinite subset of language $L$.



\null
\vfill

Theorem 3.21: L is Turing-recognizable iff some enumerator enumerates it. \\
Theorem (S42): L is Turing-recognizable iff L is enumerated by some TM. \\
Theorem (S44): L is decidable iff L is enumerable in lexicographic order
(lexicographic order has shorter strings before longer, and
alphabetic order among strings of the same length).\\

\end{document}



