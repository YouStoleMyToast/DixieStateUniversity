%
% Assignment 8d for CS3530 Computational Theory:
% Complexity
% Fall 2014
%
% Problems taken from Sipser
%

\documentclass{article}

\usepackage[margin=1in]{geometry}
\usepackage{amsfonts}
\usepackage{amsmath}
\usepackage[english]{babel}
\usepackage[utf8]{inputenc}
\usepackage{ae,aecompl}
\usepackage{enumerate}

% skip for paragraphs, don't indent
\parskip 6pt plus 1pt
\parindent=0pt
\raggedbottom

% a list environment with no bullets or numbers
\newenvironment{indentlist}{\begin{list}{}{\addtolength{\itemsep}{0.5\baselineskip}}}{\end{list}}

\begin{document}

\begin{center}
\textbf{\Large CS 3530: Assignment 8d} \\[2mm]
Fall 2014
\end{center}

\raggedright

\section*{Problems}

\subsection*{Problem 7.17 (20 points)}

\subsubsection*{Problem}

Show that, if P = NP, then every language $A\in$ P, except
$A=\emptyset$ and $A=\Sigma^*$, is NP-complete.

\subsubsection*{Solution}

Theorem 7.27(Cook-Levin):  SAT $\in$ P iff P = NP \\
Theorem 7.31: If $A \le_p B$ and $B \in$ P, then $A \in$ P \\ \ \\

According to the Cook-Levin theorem SAT is in P iff P = NP.
If we combine this with theorem 7.31 then if every language
A reduces to SAT then A is in P, it is also in NP since P = NP.
In either case since A can be verified in polynomial time
making it NP-complete. \\ 




\end{document}












