%
% Assignment 8b for CS3530 Computational Theory:
% Complexity
% Fall 2014
%
% Problems taken from Sipser
%

\documentclass{article}

\usepackage[margin=1in]{geometry}
\usepackage{amsfonts}
\usepackage{amsmath}
\usepackage[english]{babel}
\usepackage[utf8]{inputenc}
\usepackage{ae,aecompl}
\usepackage{emp,ifpdf}
\usepackage{graphicx}
\usepackage{enumerate}

\ifpdf\DeclareGraphicsRule{*}{mps}{*}{}\fi

\empaddtoTeX{\usepackage{amsmath}}
\empprelude{input boxes; input theory}

% skip for paragraphs, don't indent
\parskip 6pt plus 1pt
\parindent=0pt
\raggedbottom

% a list environment with no bullets or numbers
\newenvironment{indentlist}{\begin{list}{}{\addtolength{\itemsep}{0.5\baselineskip}}}{\end{list}}

\begin{document}
\begin{empfile}

\begin{center}
\textbf{\Large CS 3530: Assignment 8b} \\[2mm]
Fall 2014
\end{center}

\raggedright

\section*{Problems}

\subsection*{Problem 7.34 (20 points)}

\subsubsection*{Problem}

A subset of the nodes of a graph $G$ is a \textit{\textbf{dominating-set}} 
if every other node of $G$ is adjacent to some node in the subset. Let 

$$\text{\textsc{Dominating-Set}}=\{\langle G,k\rangle:
G\text{ has a dominating set with $k$ nodes}\}.$$ 

Show that it is NP-complete by giving a reduction from \textsc{Vertex-Cover}.

\subsubsection*{Solution}
Theorem 7.44: \\
$\textsc{Vertex-Cover}=\{\langle G,k\rangle|G$ is an undirected graph that has a 
$k$-node vertex cover\} is NP-complete. \\ \ \\

To show that they are equivalent we will have to do reductions to show the following: \\
1. If there is a $\textsc{Vertex-Cover}$ then there is a $\textsc{Dominating-Set}$. \\ 
2. If there is a $\textsc{Dominating-Set}$ then there a $\textsc{Vertex-Cover}$ \\ \ \\


1. Reducing from a $\textsc{Vertex-Cover}$ to a $\textsc{dominating-set}$ 

Given a vertex cover graph $G(V,E)$ with a cover size of $k$
construct a graph $G'(V,E)$ with dominating set of size $k$
by adding to each edge a new node that has edges connected 
to the original edge nodes. If there is a vertex cover for $G$
then there is a dominating set for $G'$. \\ \ \\


2. Reducing from a $\textsc{dominating-set}$ to a $\textsc{Vertex-Cover}$

The reverse of the above reduction should be possible by 
combining nodes into one of its neighbors.








\end{empfile}
\immediate\write18{mpost -tex=latex \jobname}
\end{document}



%%%%%%%%%%%%%%%%%%%%%%%%%%%%%%%%%%%%%%%%%%%%%%%%%%%%%%%%%%%%%%%%%%%%%%%%%%%%%%%%%%
%Definitions: 
%$k$: The number of cover nodes in a graph (ex: the colored ones in (S68)) \\

%vertex-cover: the $k$ nodes cover the graph if each edge ends on a colored node \\
%from site: vertex cover: every edge in G has at least one of its end points in S. \\

%Dominating-set: a subset of nodes in a graph where every node in G is adjacent to some node in the subset. \\


%%%%%%%%%%%%%%%%%%%%%%%%%%%%%%%%%%%%%%%%%%%%%%%%%%%%%%%%%%%%%%%%%%%%%%%%%%%%%%%%%%
%break a graph down into a set of nodes S where each of those nodes 
%has a set N containing every node they are connected to. If there is a 
%set D that has at least one of its nodes in every set in 
%S then it has a dominating set.

%break the graph down into a set E of edges (as pairs of nodes).
%If there is a set C that contains at least one node from
%every pair in E then there is a vertex cover. 

%%%%%%%%%%%%%%%%%%%%%%%%%%%%%%%%%%%%%%%%%%%%%%%%%%%%%%%%%%%%%%%%%%%%%%%%%%%%%%%%%%
%Hello, I was just wondering about what exactly the difference between vertex cover 
%and dominating set are. One thought I had was to break a graph down into a set of 
%nodes S where each of those nodes is a set N containing every node they are connected 
%to. From there I was thinking something along the lines of it having a dominating set 
%if there is a proposed set D that has nodes and at least one of those nodes in D can 
%be found in every set N.







