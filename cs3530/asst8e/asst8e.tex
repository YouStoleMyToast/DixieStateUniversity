%
% Assignment 8e for CS3530 Computational Theory:
% Complexity
% Fall 2014
%
% Problems taken from Sipser
%

\documentclass{article}

\usepackage[margin=1in]{geometry}
\usepackage{amsfonts}
\usepackage{amsmath}
\usepackage[english]{babel}
\usepackage[utf8]{inputenc}
\usepackage{ae,aecompl}
\usepackage{emp,ifpdf}
\usepackage{graphicx}
\usepackage{enumerate}

\ifpdf\DeclareGraphicsRule{*}{mps}{*}{}\fi

\empaddtoTeX{\usepackage{amsmath}}
\empprelude{input boxes; input theory}

% skip for paragraphs, don't indent
\parskip 6pt plus 1pt
\parindent=0pt
\raggedbottom

% a list environment with no bullets or numbers
\newenvironment{indentlist}{\begin{list}{}{\addtolength{\itemsep}{0.5\baselineskip}}}{\end{list}}

\begin{document}
\begin{empfile}

\begin{center}
\textbf{\Large CS 3530: Assignment 8e} \\[2mm]
Fall 2014
\end{center}

\raggedright

\subsection*{Problem 7.30 (20 points)}

\subsubsection*{Problem}

This problem is inspired by the single-player game
\textit{Minesweeper}, generalized to an arbitrary graph. Let $G$ be
an undirected graph, where each node either contains a single,
hidden \textit{mine} or is empty. The player chooses nodes, one by
one. If the player chooses a node containing a mine, the player
loses. If the player chooses an empty node, the player learns the
number of neighboring nodes containing mines. (A neighboring node is
one connected to the chosen node by an edge.). The player wins if
and when all empty nodes have been so chosen.

In the \textit{mine consistency problem} you are given a graph $G$,
along with numbers labeling some of $G$'s nodes. You must determine
whether a placement of mines on the remaining nodes is possible, so
that any node $v$ that is labeled $m$ has exactly $m$ neighboring
nodes containing mines. Formulate this problem as a language and
show that it is NP-complete.

\subsubsection*{Solution}

1. detrermine polynomial time verification \\ \ \\

for each node check that node v labeled m has edges connecting it to m bomb nodes \\
time is O(nodes*edges) \\ \ \\


2. reduction from a known np-complete problem to minesweeper in polynomial time\\ \ \\

To reduce \textsc{subset-sum} into minesweeper. Where \textsc{subset-sum} is: \\ \ \\

\textsc{subset-sum}=\{$\langle S, t \rangle$ | $S=\{x1,...,xj\}$ and for some 
$\{x1,...,xj\} \subseteq \{y1,...,yk\}$, we have $\Sigma yi=t$\} \\
note: the sets are multisets which allow duplicate numbers \\ \ \\

We will use this formula to determine if the sum of unknown cells
(potential bombs) is equal to (or less than) the numbers represented
from the known cells. Since a single known cells value represents more 
than one cell we will need to make sure to extract those numbers
without duplicating the count but that should be managed in polynomial 
time. \\ \ \\

If the sum is less than or equal to number of potential bombs then
we have proven that the minesweeper instance(G) is valid. \\ \ \\

Also with the reduction of \textsc{subset-sum} into minesweeperwe were 
able to show minesweeper is NP-complete since \textsc{subset-sum}
is NP-complete.

%reduce 3-SAT into minesweeper


%a-sat = (l1 & l2 & l3 & ln): is node one valid & node two....
%hiden or not = ((N | !N) & (E | !E) & (S | !S) & (W | !W))
%l1 is valid if for N, S, E, and W if at least one is not hiden and its number is <= the number of hiden 



%problem:
%all nodes have: 8,5, or 3 edges
%all nodes are: numbered or hiden (all mines are here as well as othes)

%mine consistency problem: a given grid is consistant

\end{empfile}
\immediate\write18{mpost -tex=latex \jobname}
\end{document}

%wrong-what about nodes higher than 3
%In trying to see how this problem would fit within a SAT problem I tried to break
%it down into its smallest viable component I could think of. This is where a graph 
%containing five nodes has four outer nodes that connect only to a fifth center 
%node. The four outer nodes are neighboring cells where the fifth one in the 
%center reflects the state of its neighboring cells if it is a valid configuration 
%or not. A valid cofiguration is determined by if all cells are hiden or at 
%least one outer node is not hiden and its number is less than or equal to the 
%number of outer nodes that are hiden. The whole minesweeper puzzel is valid 
%if all center nodes are valid.

%To reduce SAT into minesweeper we will use a SAT in the format of 

%hiden or not = ((N | !N) \& (E | !E) \& (S | !S) \& (W | !W))

%valid = \\(((N \& N $\le$ 3) \& !E \& !S \& !W) || ((N \& N $\le$ 2) \& ((!E \& !S) || (!S \& !W))) 
%|| ((N \& N $\le$ 1) \& (!E || !S || !W)))



